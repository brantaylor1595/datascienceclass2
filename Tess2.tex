% Options for packages loaded elsewhere
\PassOptionsToPackage{unicode}{hyperref}
\PassOptionsToPackage{hyphens}{url}
%
\documentclass[
]{article}
\usepackage{lmodern}
\usepackage{amssymb,amsmath}
\usepackage{ifxetex,ifluatex}
\ifnum 0\ifxetex 1\fi\ifluatex 1\fi=0 % if pdftex
  \usepackage[T1]{fontenc}
  \usepackage[utf8]{inputenc}
  \usepackage{textcomp} % provide euro and other symbols
\else % if luatex or xetex
  \usepackage{unicode-math}
  \defaultfontfeatures{Scale=MatchLowercase}
  \defaultfontfeatures[\rmfamily]{Ligatures=TeX,Scale=1}
\fi
% Use upquote if available, for straight quotes in verbatim environments
\IfFileExists{upquote.sty}{\usepackage{upquote}}{}
\IfFileExists{microtype.sty}{% use microtype if available
  \usepackage[]{microtype}
  \UseMicrotypeSet[protrusion]{basicmath} % disable protrusion for tt fonts
}{}
\makeatletter
\@ifundefined{KOMAClassName}{% if non-KOMA class
  \IfFileExists{parskip.sty}{%
    \usepackage{parskip}
  }{% else
    \setlength{\parindent}{0pt}
    \setlength{\parskip}{6pt plus 2pt minus 1pt}}
}{% if KOMA class
  \KOMAoptions{parskip=half}}
\makeatother
\usepackage{xcolor}
\IfFileExists{xurl.sty}{\usepackage{xurl}}{} % add URL line breaks if available
\IfFileExists{bookmark.sty}{\usepackage{bookmark}}{\usepackage{hyperref}}
\hypersetup{
  pdftitle={Test 2 Final},
  pdfauthor={Brandon Taylor},
  hidelinks,
  pdfcreator={LaTeX via pandoc}}
\urlstyle{same} % disable monospaced font for URLs
\usepackage[margin=1in]{geometry}
\usepackage{color}
\usepackage{fancyvrb}
\newcommand{\VerbBar}{|}
\newcommand{\VERB}{\Verb[commandchars=\\\{\}]}
\DefineVerbatimEnvironment{Highlighting}{Verbatim}{commandchars=\\\{\}}
% Add ',fontsize=\small' for more characters per line
\usepackage{framed}
\definecolor{shadecolor}{RGB}{248,248,248}
\newenvironment{Shaded}{\begin{snugshade}}{\end{snugshade}}
\newcommand{\AlertTok}[1]{\textcolor[rgb]{0.94,0.16,0.16}{#1}}
\newcommand{\AnnotationTok}[1]{\textcolor[rgb]{0.56,0.35,0.01}{\textbf{\textit{#1}}}}
\newcommand{\AttributeTok}[1]{\textcolor[rgb]{0.77,0.63,0.00}{#1}}
\newcommand{\BaseNTok}[1]{\textcolor[rgb]{0.00,0.00,0.81}{#1}}
\newcommand{\BuiltInTok}[1]{#1}
\newcommand{\CharTok}[1]{\textcolor[rgb]{0.31,0.60,0.02}{#1}}
\newcommand{\CommentTok}[1]{\textcolor[rgb]{0.56,0.35,0.01}{\textit{#1}}}
\newcommand{\CommentVarTok}[1]{\textcolor[rgb]{0.56,0.35,0.01}{\textbf{\textit{#1}}}}
\newcommand{\ConstantTok}[1]{\textcolor[rgb]{0.00,0.00,0.00}{#1}}
\newcommand{\ControlFlowTok}[1]{\textcolor[rgb]{0.13,0.29,0.53}{\textbf{#1}}}
\newcommand{\DataTypeTok}[1]{\textcolor[rgb]{0.13,0.29,0.53}{#1}}
\newcommand{\DecValTok}[1]{\textcolor[rgb]{0.00,0.00,0.81}{#1}}
\newcommand{\DocumentationTok}[1]{\textcolor[rgb]{0.56,0.35,0.01}{\textbf{\textit{#1}}}}
\newcommand{\ErrorTok}[1]{\textcolor[rgb]{0.64,0.00,0.00}{\textbf{#1}}}
\newcommand{\ExtensionTok}[1]{#1}
\newcommand{\FloatTok}[1]{\textcolor[rgb]{0.00,0.00,0.81}{#1}}
\newcommand{\FunctionTok}[1]{\textcolor[rgb]{0.00,0.00,0.00}{#1}}
\newcommand{\ImportTok}[1]{#1}
\newcommand{\InformationTok}[1]{\textcolor[rgb]{0.56,0.35,0.01}{\textbf{\textit{#1}}}}
\newcommand{\KeywordTok}[1]{\textcolor[rgb]{0.13,0.29,0.53}{\textbf{#1}}}
\newcommand{\NormalTok}[1]{#1}
\newcommand{\OperatorTok}[1]{\textcolor[rgb]{0.81,0.36,0.00}{\textbf{#1}}}
\newcommand{\OtherTok}[1]{\textcolor[rgb]{0.56,0.35,0.01}{#1}}
\newcommand{\PreprocessorTok}[1]{\textcolor[rgb]{0.56,0.35,0.01}{\textit{#1}}}
\newcommand{\RegionMarkerTok}[1]{#1}
\newcommand{\SpecialCharTok}[1]{\textcolor[rgb]{0.00,0.00,0.00}{#1}}
\newcommand{\SpecialStringTok}[1]{\textcolor[rgb]{0.31,0.60,0.02}{#1}}
\newcommand{\StringTok}[1]{\textcolor[rgb]{0.31,0.60,0.02}{#1}}
\newcommand{\VariableTok}[1]{\textcolor[rgb]{0.00,0.00,0.00}{#1}}
\newcommand{\VerbatimStringTok}[1]{\textcolor[rgb]{0.31,0.60,0.02}{#1}}
\newcommand{\WarningTok}[1]{\textcolor[rgb]{0.56,0.35,0.01}{\textbf{\textit{#1}}}}
\usepackage{graphicx,grffile}
\makeatletter
\def\maxwidth{\ifdim\Gin@nat@width>\linewidth\linewidth\else\Gin@nat@width\fi}
\def\maxheight{\ifdim\Gin@nat@height>\textheight\textheight\else\Gin@nat@height\fi}
\makeatother
% Scale images if necessary, so that they will not overflow the page
% margins by default, and it is still possible to overwrite the defaults
% using explicit options in \includegraphics[width, height, ...]{}
\setkeys{Gin}{width=\maxwidth,height=\maxheight,keepaspectratio}
% Set default figure placement to htbp
\makeatletter
\def\fps@figure{htbp}
\makeatother
\setlength{\emergencystretch}{3em} % prevent overfull lines
\providecommand{\tightlist}{%
  \setlength{\itemsep}{0pt}\setlength{\parskip}{0pt}}
\setcounter{secnumdepth}{-\maxdimen} % remove section numbering

\title{Test 2 Final}
\author{Brandon Taylor}
\date{6/26/2020}

\begin{document}
\maketitle

\begin{enumerate}
\def\labelenumi{\arabic{enumi}.}
\tightlist
\item
  Please clear the environment in R.
\end{enumerate}

\begin{Shaded}
\begin{Highlighting}[]
\KeywordTok{remove}\NormalTok{(}\DataTypeTok{list =} \KeywordTok{ls}\NormalTok{())}
\end{Highlighting}
\end{Shaded}

\begin{enumerate}
\def\labelenumi{\arabic{enumi}.}
\setcounter{enumi}{1}
\tightlist
\item
  Load the ``inequality'' dataset into R, and save the data frame as
  `inequality\_data'.
\end{enumerate}

\begin{Shaded}
\begin{Highlighting}[]
\KeywordTok{library}\NormalTok{(rio)}
\NormalTok{inequality_data <-}\StringTok{ }\KeywordTok{import}\NormalTok{(}\StringTok{"inequality.xlsx"}\NormalTok{)}
\KeywordTok{attach}\NormalTok{(inequality_data)}
\end{Highlighting}
\end{Shaded}

\begin{enumerate}
\def\labelenumi{\arabic{enumi}.}
\setcounter{enumi}{2}
\tightlist
\item
  Is this dataset a cross-sectional or panel dataset? Explain why in
  words and provide some R code to prove that your answer is correct.
\end{enumerate}

\begin{Shaded}
\begin{Highlighting}[]
\KeywordTok{head}\NormalTok{(inequality_data)}
\end{Highlighting}
\end{Shaded}

\begin{verbatim}
##   iso2c country inequality_gini year
## 1    AL Albania            32.9 2015
## 2    AM Armenia            32.4 2015
## 3    AT Austria            30.5 2015
## 4    BY Belarús            25.6 2015
## 5    BE Belgium            27.7 2015
## 6    BZ  Belize              NA 2015
\end{verbatim}

\begin{Shaded}
\begin{Highlighting}[]
\KeywordTok{summary}\NormalTok{(inequality_data)}
\end{Highlighting}
\end{Shaded}

\begin{verbatim}
##     iso2c             country          inequality_gini      year     
##  Length:203         Length:203         Min.   :25.40   Min.   :2015  
##  Class :character   Class :character   1st Qu.:31.55   1st Qu.:2015  
##  Mode  :character   Mode  :character   Median :35.75   Median :2015  
##                                        Mean   :36.81   Mean   :2015  
##                                        3rd Qu.:41.12   3rd Qu.:2015  
##                                        Max.   :59.10   Max.   :2015  
##                                        NA's   :123
\end{verbatim}

\begin{enumerate}
\def\labelenumi{\arabic{enumi}.}
\setcounter{enumi}{3}
\item
  Using the subset command, provide the inequality\_gini scores for
  Denmark and Sweden. library(dplyr) inequality\_datasub \textless-
  subset(inequality\_gini, select = c(``Denmark'', ``Sweden''))
\item
  Since Brazil started the Bolsa Familia conditional cash transfer
  program in 1990s, inequality in Brazil has decreased significantly.
  Just the same, inequality in Brazil is very high comparatively. Using
  the subset command, please show the inequality\_gini score for Brazil.
\end{enumerate}

inequality\_data \textless- subset(inequality\_gini, ``Brazil'')

\begin{enumerate}
\def\labelenumi{\arabic{enumi}.}
\setcounter{enumi}{5}
\tightlist
\item
  Given your answers to the previous questions, is it better to have a
  high or low inequality\_gini scores?
\end{enumerate}

It is better to have a lower score.

\begin{enumerate}
\def\labelenumi{\arabic{enumi}.}
\setcounter{enumi}{6}
\tightlist
\item
  Use the head command to get a quick peak at the data frame.
\end{enumerate}

\begin{Shaded}
\begin{Highlighting}[]
\KeywordTok{head}\NormalTok{(inequality_data)}
\end{Highlighting}
\end{Shaded}

\begin{verbatim}
##   iso2c country inequality_gini year
## 1    AL Albania            32.9 2015
## 2    AM Armenia            32.4 2015
## 3    AT Austria            30.5 2015
## 4    BY Belarús            25.6 2015
## 5    BE Belgium            27.7 2015
## 6    BZ  Belize              NA 2015
\end{verbatim}

\begin{enumerate}
\def\labelenumi{\arabic{enumi}.}
\setcounter{enumi}{7}
\item
  Write a function called ``accent.remove'' to remove the accent on
  Belarus, apply that function, and run the head command again to show
  that you removed the accent.
\item
  Sort the data by the countries with the lowest inequality\_gini scores
  and then run the head command again to show what the top 5 countries
  are.
\item
  What is the mean inequality\_gini score? Provide the relevant R code.
\end{enumerate}

\begin{Shaded}
\begin{Highlighting}[]
\KeywordTok{mean}\NormalTok{(inequality_gini)}
\end{Highlighting}
\end{Shaded}

\begin{verbatim}
## [1] NA
\end{verbatim}

\begin{enumerate}
\def\labelenumi{\arabic{enumi}.}
\setcounter{enumi}{10}
\item
  Using the ifelse command, create two new dummy variables,
  high\_inequality and low\_inequality, which takes values of either
  zero or one. The low\_inequality variable should correspond to
  countries with inequality\_gini scores below the mean. The
  high\_inequality variable should correspond to countries with
  inequality\_gini scores above the mean. (Note: we will not accept
  answers that do not use the ifelse command to create the variables.)
\item
  Runacross-tabusingthehigh\_inequalityandlow\_inequalityvariablesthatyoucre-
  ated in the previous question. The cross-tab should provide the mean
  inequality\_gini score and number of observations for each category of
  inequality. (Note: if you had trouble using the ifelse command, we
  couldn't provide points for the previous question. However, you can
  create the variables using the indexing method)
\item
  The World Bank, the African Development Bank, and the Bill and Melinda
  Gates Foundation are all working on reducing inequality in Africa.
  Write a for loop that prints the names of these three actors. (Note:
  we will not accept answers that do not provide a for loop.)
\item
  Use this website to find a variable from the World Development
  Indicators that you think is correlated with inequality. Tell us what
  variable you picked and why you picked it. (Don't worry if your
  prediction is not correct. We just want you to provide some
  rationale.)
\end{enumerate}

Mobile cellular subscriptions (per 100 people), is mobile phone
susbscriptions corllated with the inequlity level.

\begin{enumerate}
\def\labelenumi{\arabic{enumi}.}
\setcounter{enumi}{14}
\tightlist
\item
  Import that variable directly into R. (Note: if you are having
  trouble, read Mike Denly's Canvas announcement from the other day.)
\end{enumerate}

\begin{Shaded}
\begin{Highlighting}[]
\NormalTok{mobphone <-}\StringTok{ }\KeywordTok{import}\NormalTok{(}\StringTok{"API_IT.CEL.SETS.P2_DS2_en_csv_v2_1121611.csv"}\NormalTok{)}
\KeywordTok{attach}\NormalTok{(mobphone)}
\end{Highlighting}
\end{Shaded}

\begin{enumerate}
\def\labelenumi{\arabic{enumi}.}
\setcounter{enumi}{15}
\tightlist
\item
  Rename the variable that you imported into something that we can
  actually understand.
\end{enumerate}

mobphone

\begin{enumerate}
\def\labelenumi{\arabic{enumi}.}
\setcounter{enumi}{16}
\item
  Merge the new variable into the other dataset, using inequality\_data
  as the x and and your new data frame as the y. When merging use the
  command that only keeps the rows in your x data frame. Call your new
  data frame merged\_df. Ensure that you have no variables with .x or .y
  at the end of them in your new merged\_df, while at the same time
  ensuring there are still variables like country and year.
\item
  In merged\_df, remove the missing data on the basis of
  inequality\_gini and your new variable that you took from the World
  Development Indicators.
\item
  Using the filter command and piping method, only keep the data with
  inequality\_gini scores greater than 30. Save the new data frame as
  data\_greater\_30. (Note: we will not accept answers using subset.)
\item
  Using data\_greater\_30, use to R to count how many countries have the
  sequence ``ai''in their name.
\item
  Use any command from the apply family to take the sum of
  inequality\_gini in data\_greater\_30.
\item
  Label your variables in merged\_df. Any labels will suffice.
  library(labelled)
\item
  Save the labeled data frame as a Stata dataset called final\_data.
\end{enumerate}

export(merged\_df, file = ``final\_dtat.dta'')

\begin{enumerate}
\def\labelenumi{\arabic{enumi}.}
\setcounter{enumi}{23}
\tightlist
\item
  Save all of the files (i.e.~.Rmd, .dta, .xlsx, .pdf/Word Doc), push
  them to your GitHub repo, and provide us with the link to that repo.
\end{enumerate}

\url{https://github.com/brantaylor1595/datascienceclass2}

\end{document}
